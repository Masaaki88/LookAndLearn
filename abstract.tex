\documentclass[a4paper]{article}

\usepackage[utf8x]{inputenc}
\usepackage{amssymb, amsmath}
\usepackage{graphicx}
\usepackage{listings}
\usepackage{hyperref}
\usepackage{wrapfig}
\usepackage[font=small]{caption}
\usepackage{subcaption}
\usepackage{listings}
\usepackage{url}
\usepackage{bbold}
\usepackage{tipa}
\usepackage{pdfpages}

\hyphenation{simu-lated simu-la-tion ma-nipu-la-tion so-ge-nann-ten De-fi-ni-tio-nen bei-spiels-wei-se Sti-mu-lus-ei-gen-schaf-ten Ent-wick-lung-en ver-bes-sert}

\begin{document}
\pagenumbering{gobble}



\title{Look and Learn --\\A Computational Model of Gaze-Contingent Learning}
\author{Max Murakami}
\date{Zusammenfassung}

\maketitle

Diese Dissertation besch\"aftigt sich mit der Frage, mit Hilfe welcher Prinzipien sich intrinsische Motivation formalisieren l\"asst und wie diese es uns erm\"o-glichen, kausale Zusammenh\"ange unserer Umwelt zu erlernen. Konkret wird das Verhalten von sechs bis zehn Monate alten S\"auglingen untersucht, die an einem Experiment des sogenannten blickkontingenten Paradigmas teilnehmen. Hierbei lernt der S\"augling seine visuelle Umgebung mit Hilfe seines Blicks zu beeinflussen. Das Blickverhalten, das der S\"augling w\"ahrend des Lernvorgangs an den Tag legt, wird quantitativ ausgewertet und mit Hilfe von Computermodellen nachgebildet. Auf diese Weise wird ein direkter Bezug zwischen dem Verhalten des S\"auglings und mathematisch formulierten Theorien geschaffen, die Aussagen \"uber m\"ogliche zu Grunde liegende formale Prinzipien erm\"oglichen. Die Haupterkenntnis dieser Studie ist, dass sich typische Verhaltensmuster der S\"auglingsprobanden, insbesondere funktional-ausgerichtete Blickpr\"aferenzen, auf den Prozess der Vorhersageoptimierung des internen Weltmodells per aktiver Informationsmaximierung zur\"uckf\"uhren lassen.

Die Dissertation umfasst vier Kapitel. Die Einleitung umrei\ss t die wissenschaftliche Fragestellung und gibt einen \"Uberblick \"uber den Inhalt der Studie. Es folgt eine Einf\"uhrung in das Thema der intrinsischen Motivation, aus einer historischen, einer interdisziplin\"aren und einer Modellierungs-Perspektive. Entwicklungen auf der Ebene der Theorien werden ebenso abgehandelt wie wichtige Vorarbeiten und Modellierungsstudien, die einen direkten Bezug auf diese Dissertationsarbeit aufweisen. Dar\"uber hinaus wird ein konziser \"Uberblick \"uber experimentelle entwicklungspsychologische Studien zum Kontingenzlernen in S\"auglingen gegeben. Aufbauend auf den Entwicklungen der letzten Jahrzehnte wird \"ubergegangen zum experimentellen Paradigma des blickkontingenten Lernens, das 2012 von Wang und Kollegen vorgestellt wurde. Deren Studie wird im Detail diskutiert, da sie die Grundlage der empirischen Untersuchungen dieser Dissertation darstellt.

Der Methodenteil beschreibt im Detail die Materialien, Daten und Modelle, auf denen diese Studie basiert. Das Experiment wird erl\"autert ebenso wie die Datenextraktion und -verarbeitung. Die Computermodelle werden sowohl auf der konzeptionellen als auch auf der mathematischen Ebene definiert. Das Kapitel schlie\ss t mit einer Erl\"auterung des algorithmischen Modell-Fittings.

Das dritte Kapitel pr\"asentiert die Ergebnisse. Hier werden die Analysen, die statistischen Auswertungen sowie deren Interpretationen vorgestellt. Nach einer qualitativen Betrachtung und quantitativen Erfassung der Blickpr\"aferenz wird der Einfluss des Alters auf die Daten untersucht und inwiefern die Modelle Aussagen zu diesem Faktor treffen. Desweiteren wird die Dynamik des Lernverlaufs individueller Modellprobanden quantitativ charakterisiert und schlie\ss lich der Einfluss der Lerngeschwindigkeit und der Erkundungsbereitschaft auf das Lernverhalten der S\"auglinge auf unterschiedlichen Ebenen analysiert.

Im letzten Kapitel werden die Befunde der Studie zusammengefasst und vor dem Hintergrund existierender Theorien und Arbeiten kritisch diskutiert. M\"ogliche Schw\"achen des vorgestellten Forschungsansatzes, der Daten und der Analysen werden dabei ebenfalls angesprochen. Die Dissertation schlie\ss t mit einem Ausblick und Vorschl\"agen, wie der vorliegende Forschungsansatz verbessert und die Realit\"at detailgetreuer modelliert werden kann.
%
%\quad
%
%\textit{Intrinsische Motivation} ist ein Konzept, das in der Psychologie der 1950er Jahre aufkam~\cite{harlow50,harlow50b,berlyne54,white59} und sich in den letzten Jahren zu einem aktiven Forschungsthema in den Bereichen der Robotik, der k\"unstlichen Intelligenz, des maschinellen Lernens und der theoretischen Neurowissenschaften entwickelt hat~\cite{baldassarre13}. W\"ahrend Psychologen von intrinsischer Motivation sprechen, wenn etwas getan wird, weil es inh\"arent interessant oder angenehm ist~\cite{ryan00b}, existieren in der modernen algorithmisch-gepr\"agten Literatur Definitionen, die intrinsische Motivation mit Lernf\"ahigkeit assoziieren und zugleich abgrenzen von der \"Uberlebens- und Fortpflanzungsf\"ahigkeit eines Organismus~\cite{mirolli13}. Intrinsische Motivation wird in Verbindung gebracht mit Merkmalen menschlicher Zivilisation wie k\"unstlerische Kreativit\"at und wissenschaftlichem Entdeckungsdrang~\cite{ryan00,schmidhuber10}. Roboter und andere k\"unstliche Systemen mit intrinsischer Motivation auszustatten wird heutzutage als ein notwendiger Schritt zu wahrer k\"unstlicher Intelligenz angesehen~\cite{weng01}.
%
%\textit{Motivation} beschreibt \textit{warum} wir tun, was wir tun~\cite{ryan00b}, setzt dabei aber Autonomie voraus, d.h.~ein gewisses Ma\ss ~an Selbstkontrolle~\cite{mcfarland93}. Fr\"uhe Theorien der Motivation werden heutzutage als Theorien der extrinsischen Motivation bezeichnet, da sie sich auf k\"orperliche Bed\"urfnisse beschr\"anken. Ein Beispiel ist Hulls einflussreiche Theorie der Triebe, die in den 1940er und 50er Jahren formuliert wurde~\cite{hull43,hull51,hull52}. Der Begriff \textit{intrinsische Motivation} geht auf Harlow aus dem Jahr 1950 zur\"uck. Harlow berichtete damals von Affen, die sich in einer Art und Weise verhielten, die sich nicht mit den damaligen Theorien der Motivation erkl\"aren lie\ss~\cite{harlow50}. Berlyne schlug daraufhin vor, intrinsische Motivation mittels bestimmter Stimulus-Eigenschaften wie Neuartigkeit und Vorhersagbarkeit zu erkl\"aren~\cite{berlyne54,berlyne60,berlyne71}. Da sich diese Eigenschaften auf das Wissen des jeweiligen Subjekts beziehen, formulierte er damit die \textit{wissensbasierte} Sichtweise der intrinsischen Motivation~\cite{oudeyer07}. Demgegen\"uber steht die \textit{kompetenzbasierte} Sichtweise, die von White formuliert wurde und besagt, dass die Steigerung der Kompetenz mit der Umwelt zu interagieren im Zentrum der intrinsischen Motivation steht~\cite{white59}.
%
%Die meisten Modelle der intrinsischen Motivation sind wissensbasiert, wie beispielsweise Schmidhubers bahnbrechende Arbeit von 1991, in der er Neugier als Drang zur Vorhersageoptimierung mittels Informationsmaximierung implementierte~\cite{schmidhuber91}. Jedoch existieren auch kompetenzbasierte Modelle, die \"ublicherweise das Erlernen und Perfektionieren bestimmter Fertigkeiten thematisieren, so wie Bartos Intrinsically Motivated Reinforcement Learning von 2004~\cite{barto04,singh05}. Zwar ist man sich dar\"uber einig, dass mit beiden Ans\"atzen sowohl das Wissen als auch die Kontrolle des Agenten \"uber die Umwelt zunimmt, welcher dieser Aspekte jedoch das fundamentale Prinzip ist, wird kontrovers diskutiert~\cite{mirolli13,barto13,schmidhuber09}.
%
%Nach neurowissenschaftlichem Erkenntnisstand basiert die intrinsische Motivation im Gehirn auf wissensbasierten Signalen, wobei dem Neurotransmitter \textit{Dopamin} eine zentrale Rolle zugesprochen wird, da er sowohl unerwartete als auch neuartige Stimuluseigenschaften zu kodieren scheint~\cite{dommett05,horvitz00,lisman05,otmakova12}. Kompetenzbasierte Signale wurden hingegen noch nicht gefunden~\cite{mirolli13}. Basierend auf diesen Erkenntnissen formulierten Redgrave und Gurney ihre Hypothese des \textit{Repetition Bias}, wonach es einen kausalen Zusammenhang zwischen intrinsisch motiviertem Kontingenzlernen und der Bildung von Verhaltensmustern gibt~\cite{redgrave06,gurney12,redgrave12}.
%
%In einer abschlie\ss enden Betrachtung ist anzumerken, dass aus einer evolution\"aren Perspektive die Grenze zwischen extrinsischer und intrinsischer Motivation verschwimmt~\cite{barto13}. Barto pl\"adiert daf\"ur, diese strikte Unterscheidung aufzuheben und stattdessen von einem Kontinuum zu sprechen, wobei sich am einen Ende des Spektrums extrinsische Motivation eher direkt und unmittelbar auf evolution\"aren Erfolg auswirkt, auf der anderen Seite der Einfluss intrinsischer Motivation entsprechend eher indirekt und weniger offensichtlich ist~\cite{barto13}.
%
%\quad
%
%Untersuchungen des \textit{Kontingenzlernens} in S\"auglingen stellen wegen derer langsamen motorischen Entwicklung eine besondere Herausforderung dar, weshalb sie sich traditionell auf Reflexe und basales motorisches Verhalten beschr\"anken~\cite{rovee80,kalnins73}. Beispielsweise wurde der Saugreflex benutzt um zu zeigen, dass bereits Neugeborene ihr Saugverhalten \"andern, wenn sie damit Eigenschaften von Stimuli beeinflussen~\cite{decasper80}. In einem anderen Fall fanden Rochat und Striano anhand des Suchreflexes heraus, dass Neugeborene zwischen Ber\"uhrungen durch sich selbst und Ber\"uhrungen durch Objekte unterscheiden k\"onnen~\cite{rochat00}. In der sogenannten Mobile-Aufgabe zeigten Rovee-Collier und Kollegen, dass zwei Monate alte S\"auglinge die Kontingenz zwischen ihrem Strampeln und der Bewegung eines mit ihren Beinen verbundenen Mobiles erlernen~\cite{rovee80,rovee01}. Weiterhin konnte Kenward zeigen, dass zehn Monate alte S\"auglinge visuelle Stimuli antizipieren, die sie per Knopfdruck ausl\"osen, womit sie die Formierung von Erwartungen und Vorhersagen demonstrierten~\cite{kenward10}.
%
%Da sich die Augenkoordination bereits relativ fr\"uh entwickelt~\cite{bronson90,johnson91}, etablierte sich die Analyse des Blickverhaltens als informatives, flexibel einsetzbares Mittel in der S\"auglingsforschung~\cite{fantz64,cohen72,baillargeon85,haith88,quinn93,csibra99,smith03,mcmurray04,aslin07}. Dank technischer Neuerungen erm\"oglicht die \textit{Blickerfassungsmethode} (eye tracking) automatisierte und pr\"azise Auswertungen des Blickverhaltens in Echtzeit und geh\"ort dadurch zum Repertoire moderner S\"auglingsforschung~\cite{johnson03,mcmurray04,gredebaeck09,oakes12,aslin12}. Durch die Blickerfassungsmethode kann eine neue Klasse von Experimenten realisiert werden, die sogenannten \textit{blickkontingenten} Experimente, die zwar schon seit geraumer Zeit mit Erwachsenen durchgef\"uhrt werden~\cite{reader73,duchowski02}, aber erst seit wenigen Jahren in der S\"auglingsforschung angewendet werden~\cite{holmboe08,deligianna11,wass11,tummeltshammer14,miyazaki14}.
%
%Ein spezieller Fall ist das \textit{blickkontingente Paradigma} (gaze-contingent paradigm), das von Wang und Kollegen 2012 vorgestellt wurde~\cite{wang12} und die Grundlage f\"ur die empirischen Untersuchungen in dieser Dissertation darstellt. In der Originalversion dieses Experiments ist eine rote Scheibe auf dem Bildschirm zu sehen, die bei Fixation darauf das Erscheinen eines Tierbilds ausl\"ost und somit als optischer Schalter fungiert.
%In der Originalstudie wurde gezeigt, dass sechs und acht Monate alte S\"auglinge lernen das Tierbild zu antizipieren, nachdem sie es ausl\"osen, und folglich die Kontingenz zwischen ihrem Blick und der visuellen Antwort erfolgreich erwerben. Mit Hilfe der erweiterten Zwei-Scheiben-Version demonstrierten die Autoren, dass die S\"auglinge tats\"achlich das Erscheinen des Tierbilds mit ihrem Blick bezwecken, da sie den optischen Schalter gegen\"uber einer identischen Scheibe ohne Schalterfunktion pr\"aferieren. Interessanterweise konnte ein Gro\ss teil der erwachsenen Probanden, die das Experiment ebenfalls absolvierten, den Mechanismus nicht erkl\"aren; diejenigen, die es konnten, zeigten ein vergleichbares Blickverhalten wie die S\"auglinge, was f\"ur einen Erwerb von Einsicht und Erwartungen in den S\"auglingen spricht.
%
%\quad
%
%Das Experiment, das im Rahmen dieser Dissertation ausgewertet wurde, basiert auf der Zwei-Scheiben-Version des blickkontingenten Paradigmas von Wang et al. Im Gegensatz zur Originalstudie wurden jedoch zum einen sechs, acht und auch zehn Monate alte S\"auglinge getestet, zum anderen wurde der Test f\"ur jeden Probanden zwei Mal wiederholt, das erste Mal nach 15 bis 19 Minuten, das zweite Mal nach einer Woche. Au\ss erdem gab es eine experimentelle Kontrollbedingung, in der Probanden keine Tierbilder per Blick ausl\"osen konnten, sondern lediglich ein Video sahen vom Bildschirm, das bei einem fr\"uheren Test eines S\"auglings in der aktiven Bedingung aufgezeichnet wurde. Im Rahmen dieser Studie wurden ausschlie\ss lich die Daten der aktiven, blickkontingenten Gruppe ausgewertet und modelliert.
%
%Die Rohdaten des Eye-Trackers wurden sowohl in der zeitlichen als auch der r\"aumlichen Dom\"ane gefiltert bzw.~ausgewertet. Zum einen wurden Zeitabschnitte ausgeschlossen, in denen die Probanden nicht aktiv am Experiment teilnahmen. Zum anderen wurde mittels einer Post-Hoc-Sch\"atzung sichergestellt, dass nur Tests in die Analyse aufgenommen wurden, in denen die Blickkontingenz mit einer hohen Zuverl\"assigkeit gew\"ahrleistet war und die Lernaufgabe nicht durch Kalibrierungsungenauigkeiten stark erschwert wurde.
%
%Modell~1 basiert auf einer Studie, in der die G\"ultigkeit der Repetition-Bias-Hypothese \"uberpr\"uft werden sollte~\cite{bg13}. Da diese Hypothese Aussagen \"uber Verhaltensanpassung in intrinsisch motivierten Lernsituationen trifft, ist das Modell relevant f\"ur diese Studie und wurde \"ubernommen und an das blickkontingente Lernexperiment angepasst. Es umfasst zum einen ein Modell der neuronalen Informationsverarbeitung in relevanten Hirnregionen, insbesondere den Basalganglien und assoziierten Bereichen. Zum anderen enth\"alt es ein Vorhersagesystem, das den Wissensstand des Probanden repr\"asentiert, sich den Erfahrungen des Probanden anpasst, ma\ss geblich das Verhalten zu Gunsten der Informationsmaximierung beeinflusst und die Aussch\"uttung von Dopamin als sensorischem Vorhersagefehler veranlasst. Letzteres moduliert die synaptische Plastizit\"at zwischen Hirnrinde und Basalganglien und tr\"agt somit zur Verhaltensanpassung bei. 
%Modell~2 ist eine abstrahierte Version von Modell~1, in der der biologisch motivierte Teil weggelassen wurde. Somit besteht es einzig aus dem Vorhersagesystem, das das Wissen und die intrinsische Motivation des Probanden repr\"asentiert und sich somit auf die abstrakten Prinzipien der intrinsisch motivierten Verhaltenssteuerung beschr\"ankt. Beide Modelle werden mit Hilfe des CMA-ES-Algorithmus gefittet, wobei der Unterschied zwischen dem simulierten und dem experimentell beobachteten Verhalten pro Proband in einer stochastischen Optimierung minimiert wird~\cite{hansen06}.
%
%\quad
%
%Die Simulationsergebnisse zeigen, dass beide Modelle die auf die funktionale Scheibe ausgerichtete Blickpr\"aferenz der S\"auglinge reproduzieren k\"onnen. Zudem replizieren sie den experimentellen Befund, dass die Blickpr\"aferenz der acht und zehn Monate alten S\"auglinge im Gegensatz zu den sechs Monate alten stark ausgepr\"agt ist. Die Datenlage bez\"uglich der Modelle l\"asst allerdings keine statistisch haltbare Erkl\"arung zu.
%
%Auf der Ebene der einzelnen Probanden konnten zeitlich unterschiedlich ausgepr\"agte Lernverl\"aufe beobachtet werden, die zu qualitativ unterschiedlichen Verhaltensmustern f\"uhren. Dies lieferte einen ersten Hinweis darauf, dass die Lerngeschwindigkeit, also die Adaptionsrate des Vorhersagesystems, ma\ss geblich die Entstehung der Blickpr\"aferenz beeinflusst. Verschiedene Analysen f\"uhrten zu folgendem Ergebnis: Je langsamer ein Proband lernt, desto sp\"ater erwirbt er die Kontingenz, desto st\"arker ausgepr\"agt ist seine Blickpr\"aferenz. Der Hauptbefund l\"asst sich somit wie folgt formulieren: \textit{Die funktional-ausgerichtete Blickpr\"aferenz ist eine Konsequenz der andauernden Vorhersageoptimierung.} Allgemeiner ausgedr\"uckt: \textit{Fortw\"ahrende Informationsmaximierung f\"uhrt zur Bildung von Verhaltenspr\"aferenzen.} 
%Dieses Ergebnis steht im Einklang mit der Repetition-Bias-Hypothese, wonach Verhaltenspr\"aferenzen auftreten m\"ussen, damit kontingente, noch nicht erworbene Konsequenzen des eigenen Handelns zuverl\"assig erlernt werden k\"onnen~\cite{redgrave06,gurney12,redgrave12}. Dieser Sichtweise zufolge ist die Verhaltenspr\"aferenz eine vor\"ubergehende Phase der Vorhersageoptimierung, die solange anh\"alt, bis die Vorhersagen des internen Modells den Erfahrungen bez\"uglich der Interaktion mit der Umwelt entsprechen.
%
%Desweiteren ergaben die Modellanalysen, dass langsame Lerner weniger zum Explorieren neigen als schnelle Lerner. Dieses Ph\"anomen l\"asst sich gut mit Hilfe des \textit{exploration-exploitation}-Dilemmas verstehen, das in der Literatur des verst\"arkenden Lernens seit l\"angerem bekannt ist~\cite{sutton98}. Dieses Dilemma besagt, dass ein Agent, der ein Belohnungssignal maximieren will, sich zu jeder Zeit entscheiden muss, ob er sein Wissen ausnutzen sollte um eine bekannte Menge an Belohnung zu erhalten (exploitation), oder ob er lieber noch unbekannte Optionen ausprobieren sollte, um eine potentiell gr\"o\ss ere Menge an Belohnung zu erhalten (exploration). Im Falle des intrinsisch motivierten Kontingenzlernens ist die Belohnung der erwartete Informationsgehalt, den der Agent nach Ausf\"uhrung einer Aktion erh\"alt, basierend auf seinem Wissensstand. Der Repetition-Bias-Hypothese zufolge ist die erwartete Belohnung gering am Anfang und am Ende des Kontingenzlernens. Zu diesen Zeiten ist das Explorieren also wichtig, um Aktionen mit unerwarteten Konsequenzen zu entdecken. Sobald eine solche gefunden wurde, herrscht die Exploitation-Phase vor, in der stetig das Vorhersagemodell bez\"uglich dieser Aktion verbessert wird. Je l\"anger diese Phase also dauert, wie es bei langsamen Lernern der Fall ist, desto weniger ist der Agent auf Exploration angewiesen. Auf der anderen Seite haben schnelle Lerner ein gr\"o\ss eres Bed\"urfnis zum Explorieren, da die Exploitation-Phase k\"urzer ist und schneller wieder nach neuen erlernbaren Kontingenzen gesucht wird.


\end{document}